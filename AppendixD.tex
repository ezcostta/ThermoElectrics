\section{Appendix D}

Next, we move forward to determine the equations appearing in Appendix D in the paper, the currents for nonisothermal conditions. We still consider the more general case in which the quaisparticles is included.

\subsection{Equations (D1), (D2) and (D3)}
For equations (D1), (D2) and (D3) the electrode $L$ is taken as the probe. By using Eq. \eqref{JL:general:new},
\begin{multline*}
J_{L}=
\mathcal{L}_{LR,\mu}^{ET}(\delta\mu_{L}-\delta\mu_{R})+
\mathcal{L}_{LR,\mu}^{CAR}\left(\delta\mu_{L}+\delta\mu_{R}\right)
+
(\mathcal{L}_{LR,T}^{ET}+\mathcal{L}_{LR,T}^{CAR})(\delta T_{L}-\delta T_{R})
+
2\mathcal{L}_{LL,\mu}^{DAR}\delta\mu_{L}
\\+
\mathcal{L}^{QP}_{LS,\mu}(\delta\mu_{L}-\delta\mu_{S})
+
\mathcal{L}^{QP}_{LS,T}(\delta T_{L}-\delta T_{S}).
\end{multline*}

By setting $J_{L}=0$, $\Delta T_{LS}=\delta T_{L}-\delta T_{S}$, $\Delta T_{LR}=\delta T_{L}-\delta T_{R}$, $\delta\mu_{S}=0$, we obtain:
\begin{multline*}
0=
\mathcal{L}_{LR,\mu}^{ET}(\delta\mu_{L}-\delta\mu_{R})+
\mathcal{L}_{LR,\mu}^{CAR}\left(\delta\mu_{L}+\delta\mu_{R}\right)
+
(\mathcal{L}_{LR,T}^{ET}+\mathcal{L}_{LR,T}^{CAR})\Delta T_{LR}
+
2\mathcal{L}_{LL,\mu}^{DAR}\delta\mu_{L}
\\+
\mathcal{L}^{QP}_{LS,\mu}\delta\mu_{L}
+
\mathcal{L}^{QP}_{LS,T}\Delta T_{LS},
\end{multline*}
and by grouping the similar terms we have:
\begin{multline*}
0=
(\mathcal{L}_{LR,\mu}^{ET}+\mathcal{L}_{LR,\mu}^{CAR}+2\mathcal{L}_{LL,\mu}^{DAR}+\mathcal{L}^{QP}_{LS,\mu})\delta\mu_{L}
+
(\mathcal{L}_{LR,\mu}^{CAR}-\mathcal{L}_{LR,\mu}^{ET})\delta\mu_{R}
\\+
(\mathcal{L}_{LR,T}^{ET}+\mathcal{L}_{LR,T}^{CAR})\Delta T_{LR}
+
\mathcal{L}^{QP}_{LS,T}\Delta T_{LS},
\end{multline*}
which allows us express $\delta\mu_{R}$ in terms of $\delta\mu_{L}$ and the temperature differences. Thus, we have:
\begin{multline*}
\delta\mu_{R}=\dfrac{
(\mathcal{L}_{LR,\mu}^{ET}+\mathcal{L}_{LR,\mu}^{CAR}+2\mathcal{L}_{LL,\mu}^{DAR}+\mathcal{L}^{QP}_{LS,\mu})\delta\mu_{L}
+
(\mathcal{L}_{LR,T}^{ET}+\mathcal{L}_{LR,T}^{CAR})\Delta T_{LR}
+
\mathcal{L}^{QP}_{LS,T}\Delta T_{LS}}{\mathcal{L}_{LR,\mu}^{ET}-\mathcal{L}_{LR,\mu}^{CAR}},
\end{multline*}
and it is convenient to split the chemical and thermal differences, so we have:
\begin{multline}\label{deltamuR:D1}
\delta\mu_{R}=\dfrac{
(\mathcal{L}_{LR,\mu}^{ET}+\mathcal{L}_{LR,\mu}^{CAR}+2\mathcal{L}_{LL,\mu}^{DAR}+\mathcal{L}^{QP}_{LS,\mu})
}{\mathcal{L}_{LR,\mu}^{ET}-\mathcal{L}_{LR,\mu}^{CAR}}\delta\mu_{L}
+
\dfrac{
(\mathcal{L}_{LR,T}^{ET}+\mathcal{L}_{LR,T}^{CAR})\Delta T_{LR}
+
\mathcal{L}^{QP}_{LS,T}\Delta T_{LS}}{\mathcal{L}_{LR,\mu}^{ET}-\mathcal{L}_{LR,\mu}^{CAR}}
\end{multline}
where we have exchanged the order of the terms in the denominator to get ride of the minus signal. 


\subsubsection{Eq. (D1)}

The current $J_{RS}$ is given by Eq. \eqref{JR:general},
\begin{multline*}
J_{RS}=
\mathcal{L}_{RL,\mu}^{ET}(\delta\mu_{R}-\delta\mu_{L})+
\mathcal{L}_{RL,\mu}^{CAR}\left(\delta\mu_{R}+\delta\mu_{L}\right)
+
(\mathcal{L}_{RL,T}^{ET}+\mathcal{L}_{RL,T}^{CAR})(\delta T_{R}-\delta T_{L})
+
2\mathcal{L}_{RR,\mu}^{DAR}\delta\mu_{R}
\\+
\mathcal{L}^{QP}_{RS,\mu}(\delta\mu_{R}-\delta\mu_{S})
+
\mathcal{L}^{QP}_{RS,T}(\delta T_{R}-\delta T_{S}).
\end{multline*}
which may be written as follows:
\begin{multline*}
J_{RS}=
(\mathcal{L}_{RL,\mu}^{ET}+\mathcal{L}_{RL,\mu}^{CAR}+2\mathcal{L}_{RR,\mu}^{DAR}+\mathcal{L}^{QP}_{RS,\mu})\delta\mu_{R}
+
(\mathcal{L}_{RL,\mu}^{CAR}-\mathcal{L}_{RL,\mu}^{ET})\delta\mu_{L}
\\+
(\mathcal{L}_{RL,T}^{ET}+\mathcal{L}_{RL,T}^{CAR})\Delta T_{RL}
+
\mathcal{L}^{QP}_{RS,T}\Delta T_{RS}.
\end{multline*}

Next, we substitute the expression for $\delta\mu_{R}$ which leads to
\begin{multline*}
J_{RS}=
(\mathcal{L}_{RL,\mu}^{ET}+\mathcal{L}_{RL,\mu}^{CAR}+2\mathcal{L}_{RR,\mu}^{DAR}+\mathcal{L}^{QP}_{RS,\mu})
\left[\dfrac{
(\mathcal{L}_{LR,\mu}^{ET}+\mathcal{L}_{LR,\mu}^{CAR}+2\mathcal{L}_{LL,\mu}^{DAR}+\mathcal{L}^{QP}_{LS,\mu})
}{\mathcal{L}_{LR,\mu}^{ET}-\mathcal{L}_{LR,\mu}^{CAR}}\delta\mu_{L}
\right.\\\left.+
\dfrac{
(\mathcal{L}_{LR,T}^{ET}+\mathcal{L}_{LR,T}^{CAR})\Delta T_{LR}
+
\mathcal{L}^{QP}_{LS,T}\Delta T_{LS}}{\mathcal{L}_{LR,\mu}^{ET}-\mathcal{L}_{LR,\mu}^{CAR}}\right]
+
(\mathcal{L}_{RL,\mu}^{CAR}-\mathcal{L}_{RL,\mu}^{ET})\delta\mu_{L}
\\+
(\mathcal{L}_{RL,T}^{ET}+\mathcal{L}_{RL,T}^{CAR})\Delta T_{RL}
+
\mathcal{L}^{QP}_{RS,T}\Delta T_{RS},
\end{multline*}
which may be written as follows:
\begin{multline*}
J_{RS}=
\dfrac{
    (\mathcal{L}_{LR,\mu}^{ET}+\mathcal{L}_{LR,\mu}^{CAR}+2\mathcal{L}_{RR,\mu}^{DAR}+\mathcal{L}^{QP}_{RS,\mu})
    (\mathcal{L}_{LR,\mu}^{ET}+\mathcal{L}_{LR,\mu}^{CAR}+2\mathcal{L}_{LL,\mu}^{DAR}+\mathcal{L}^{QP}_{LS,\mu})
}{\mathcal{L}_{LR,\mu}^{ET}-\mathcal{L}_{LR,\mu}^{CAR}}\delta\mu_{L}+
(\mathcal{L}_{RL,\mu}^{CAR}-\mathcal{L}_{RL,\mu}^{ET})\delta\mu_{L}
\\+
(\mathcal{L}_{RL,\mu}^{ET}+\mathcal{L}_{RL,\mu}^{CAR}+2\mathcal{L}_{RR,\mu}^{DAR}+\mathcal{L}^{QP}_{RS,\mu})
\left[\dfrac{
(\mathcal{L}_{LR,T}^{ET}+\mathcal{L}_{LR,T}^{CAR})\Delta T_{LR}
+
\mathcal{L}^{QP}_{LS,T}\Delta T_{LS}}{\mathcal{L}_{LR,\mu}^{ET}-\mathcal{L}_{LR,\mu}^{CAR}}\right]
\\+
(\mathcal{L}_{RL,T}^{ET}+\mathcal{L}_{RL,T}^{CAR})\Delta T_{RL}
+
\mathcal{L}^{QP}_{RS,T}\Delta T_{RS},
\end{multline*}
and putting all the terms of common denominator, 
\begin{multline}\label{JRSD1:intermed:eq1}
J_{RS}=\left[
\dfrac{
(\mathcal{L}_{LR,\mu}^{ET}+\mathcal{L}_{LR,\mu}^{CAR}+2\mathcal{L}_{RR,\mu}^{DAR}+\mathcal{L}^{QP}_{RS,\mu})
(\mathcal{L}_{LR,\mu}^{ET}+\mathcal{L}_{LR,\mu}^{CAR}+2\mathcal{L}_{LL,\mu}^{DAR}+\mathcal{L}^{QP}_{LS,\mu})
-(\mathcal{L}_{RL,\mu}^{CAR}-\mathcal{L}_{RL,\mu}^{ET})^{2}
}
{\mathcal{L}_{LR,\mu}^{ET}-\mathcal{L}_{LR,\mu}^{CAR}}\right]\delta\mu_{L}
\\+
(\mathcal{L}_{RL,\mu}^{ET}+\mathcal{L}_{RL,\mu}^{CAR}+2\mathcal{L}_{RR,\mu}^{DAR}+\mathcal{L}^{QP}_{RS,\mu})
\left[\dfrac{
(\mathcal{L}_{LR,T}^{ET}+\mathcal{L}_{LR,T}^{CAR})\Delta T_{LR}
+
\mathcal{L}^{QP}_{LS,T}\Delta T_{LS}}{\mathcal{L}_{LR,\mu}^{ET}-\mathcal{L}_{LR,\mu}^{CAR}}\right]
\\+
(\mathcal{L}_{LR,\mu}^{ET}-\mathcal{L}_{LR,\mu}^{CAR})\left[\dfrac{(\mathcal{L}_{RL,T}^{ET}+\mathcal{L}_{RL,T}^{CAR})\Delta T_{RL}
+
\mathcal{L}^{QP}_{RS,T}\Delta T_{RS}}{\mathcal{L}_{LR,\mu}^{ET}-\mathcal{L}_{LR,\mu}^{CAR}}\right].
\end{multline}

The numerator appearing in Eq. \eqref{JRSD1:intermed:eq1} may be expressed as follows:
\begin{multline*}
(\mathcal{L}_{LR,\mu}^{ET}+\mathcal{L}_{LR,\mu}^{CAR}+2\mathcal{L}_{RR,\mu}^{DAR}+\mathcal{L}^{QP}_{RS,\mu})
(\mathcal{L}_{LR,\mu}^{ET}+\mathcal{L}_{LR,\mu}^{CAR}+2\mathcal{L}_{LL,\mu}^{DAR}+\mathcal{L}^{QP}_{LS,\mu})
-(\mathcal{L}_{RL,\mu}^{CAR}-\mathcal{L}_{RL,\mu}^{ET})^{2}
\\=
(\mathcal{L}_{LR,\mu}^{ET}+\mathcal{L}_{LR,\mu}^{CAR})^{2}
-
(\mathcal{L}_{RL,\mu}^{CAR}-\mathcal{L}_{RL,\mu}^{ET})^{2}
\\+
(\mathcal{L}_{LR,\mu}^{ET}+\mathcal{L}_{LR,\mu}^{CAR})
(2\mathcal{L}_{LL,\mu}^{DAR}+\mathcal{L}^{QP}_{LS,\mu})
+
(\mathcal{L}_{LR,\mu}^{ET}+\mathcal{L}_{LR,\mu}^{CAR})
(2\mathcal{L}_{RR,\mu}^{DAR}+\mathcal{L}^{QP}_{RS,\mu})
\\+
(2\mathcal{L}_{RR,\mu}^{DAR}+\mathcal{L}^{QP}_{RS,\mu})
(2\mathcal{L}_{LL,\mu}^{DAR}+\mathcal{L}^{QP}_{LS,\mu})
\end{multline*}
which leads to 
\begin{multline*}
(\mathcal{L}_{LR,\mu}^{ET}+\mathcal{L}_{LR,\mu}^{CAR}+2\mathcal{L}_{RR,\mu}^{DAR}+\mathcal{L}^{QP}_{RS,\mu})
(\mathcal{L}_{LR,\mu}^{ET}+\mathcal{L}_{LR,\mu}^{CAR}+2\mathcal{L}_{LL,\mu}^{DAR}+\mathcal{L}^{QP}_{LS,\mu})
-(\mathcal{L}_{RL,\mu}^{CAR}-\mathcal{L}_{RL,\mu}^{ET})^{2}
\\=
4\mathcal{L}_{LR,\mu}^{ET}\mathcal{L}_{LR,\mu}^{CAR}
+
(\mathcal{L}_{LR,\mu}^{ET}+\mathcal{L}_{LR,\mu}^{CAR})
(2\mathcal{L}_{LL,\mu}^{DAR}+2\mathcal{L}_{RR,\mu}^{DAR})
\\+
(\mathcal{L}_{LR,\mu}^{ET}+\mathcal{L}_{LR,\mu}^{CAR})
(\mathcal{L}^{QP}_{LS,\mu}+\mathcal{L}^{QP}_{RS,\mu})
+
(2\mathcal{L}_{RR,\mu}^{DAR}+\mathcal{L}^{QP}_{RS,\mu})
(2\mathcal{L}_{LL,\mu}^{DAR}+\mathcal{L}^{QP}_{LS,\mu})
\end{multline*}
and with some algebra we write,
\begin{multline*}
(\mathcal{L}_{LR,\mu}^{ET}+\mathcal{L}_{LR,\mu}^{CAR}+2\mathcal{L}_{RR,\mu}^{DAR}+\mathcal{L}^{QP}_{RS,\mu})
(\mathcal{L}_{LR,\mu}^{ET}+\mathcal{L}_{LR,\mu}^{CAR}+2\mathcal{L}_{LL,\mu}^{DAR}+\mathcal{L}^{QP}_{LS,\mu})
-(\mathcal{L}_{RL,\mu}^{CAR}-\mathcal{L}_{RL,\mu}^{ET})^{2}
\\=
\mathcal{L}_{LR,\mu}^{ET}
(2\mathcal{L}_{LL,\mu}^{DAR}+4\mathcal{L}_{LR,\mu}^{CAR}+2\mathcal{L}_{RR,\mu}^{DAR})
+
\mathcal{L}_{LR,\mu}^{CAR}
(2\mathcal{L}_{LL,\mu}^{DAR}+2\mathcal{L}_{RR,\mu}^{DAR})
+
4\mathcal{L}_{LL,\mu}^{DAR}\mathcal{L}_{RR,\mu}^{DAR}
\\+
(\mathcal{L}_{LR,\mu}^{ET}+\mathcal{L}_{LR,\mu}^{CAR})
(\mathcal{L}^{QP}_{LS,\mu}+\mathcal{L}^{QP}_{RS,\mu})
+
2\mathcal{L}_{RR,\mu}^{DAR}\mathcal{L}^{QP}_{LS,\mu}
+
2\mathcal{L}_{LL,\mu}^{DAR}\mathcal{L}^{QP}_{RS,\mu}
+
\mathcal{L}^{QP}_{LS,\mu}\mathcal{L}^{QP}_{RS,\mu}
\end{multline*}
and by comparing this expression, we see the right hand side is equal to Eq. \eqref{D:definition} multiplied by 2. In this way, we write:
\begin{align*}
(\mathcal{L}_{LR,\mu}^{ET}+\mathcal{L}_{LR,\mu}^{CAR}+2\mathcal{L}_{RR,\mu}^{DAR}+\mathcal{L}^{QP}_{RS,\mu})
(\mathcal{L}_{LR,\mu}^{ET}+\mathcal{L}_{LR,\mu}^{CAR}+2\mathcal{L}_{LL,\mu}^{DAR}+\mathcal{L}^{QP}_{LS,\mu})
-(\mathcal{L}_{RL,\mu}^{CAR}-\mathcal{L}_{RL,\mu}^{ET})^{2}
=2D.
\end{align*}

By substituting the expression above into Eq. \eqref{JRSD1:intermed:eq1}
\begin{multline}\label{JRSD1:intermed:eq2}
J_{RS}=\left[
\dfrac{
2D
}
{\mathcal{L}_{LR,\mu}^{ET}-\mathcal{L}_{LR,\mu}^{CAR}}\right]\delta\mu_{L}
\\+
(\mathcal{L}_{RL,\mu}^{ET}+\mathcal{L}_{RL,\mu}^{CAR}+2\mathcal{L}_{RR,\mu}^{DAR}+\mathcal{L}^{QP}_{RS,\mu})
\left[\dfrac{
(\mathcal{L}_{LR,T}^{ET}+\mathcal{L}_{LR,T}^{CAR})\Delta T_{LR}
+
\mathcal{L}^{QP}_{LS,T}\Delta T_{LS}}{\mathcal{L}_{LR,\mu}^{ET}-\mathcal{L}_{LR,\mu}^{CAR}}\right]
\\+
(\mathcal{L}_{LR,\mu}^{ET}-\mathcal{L}_{LR,\mu}^{CAR})\left[\dfrac{(\mathcal{L}_{RL,T}^{ET}+\mathcal{L}_{RL,T}^{CAR})\Delta T_{RL}
+
\mathcal{L}^{QP}_{RS,T}\Delta T_{RS}}{\mathcal{L}_{LR,\mu}^{ET}-\mathcal{L}_{LR,\mu}^{CAR}}\right],
\end{multline}

Eq. \eqref{JRSD1:intermed:eq2} can be written in a more compact form by using the set of equations of appendix C. In fact, we have:
\begin{align*}
\mathcal{L}_{LR,\mu}^{ET}+\mathcal{L}_{LR,\mu}^{CAR}+2\mathcal{L}_{LL,\mu}^{DAR}+\mathcal{L}^{QP}_{LS,\mu}
&=
2eDR_{RS,RS}
\\
\mathcal{L}_{LR,\mu}^{ET}-\mathcal{L}_{LR,\mu}^{CAR}=2eDR_{RS,LS}
\end{align*}
which allows us to write, 
\begin{align*}
\dfrac{\mathcal{L}_{LR,\mu}^{ET}+\mathcal{L}_{LR,\mu}^{CAR}+2\mathcal{L}_{LL,\mu}^{DAR}+\mathcal{L}^{QP}_{LS,\mu}}{\mathcal{L}_{LR,\mu}^{ET}-\mathcal{L}_{LR,\mu}^{CAR}}
=
\dfrac{2eDR_{RS,RS}}{2eDR_{RS,LS}}
=
\dfrac{R_{RS,RS}}{R_{RS,LS}}.
\end{align*}
and the first term of Eq. \eqref{JRSD1:intermed:eq2} is expressed in terms of $R_{RS,LS}$ as well. In fact, we write Eq. \eqref{JRSD1:intermed:eq2} as follows:
\begin{multline*}
J_{RS}=
\dfrac{\delta\mu_{L}}{eR_{RS,LS}}
+
\dfrac{R_{RS,RS}}{R_{RS,LS}}[
(\mathcal{L}_{LR,T}^{ET}+\mathcal{L}_{LR,T}^{CAR})\Delta T_{LR}
+
\mathcal{L}^{QP}_{LS,T}\Delta T_{LS}]
\\+
(\mathcal{L}_{RL,T}^{ET}+\mathcal{L}_{RL,T}^{CAR})\Delta T_{RL}
+
\mathcal{L}^{QP}_{RS,T}\Delta T_{RS},
\end{multline*}
or
\begin{align}\label{JRSD1:intermed:eq4}
J_{RS}=
\dfrac{\delta\mu_{L}}{eR_{RS,LS}}
+
\left[\dfrac{R_{RS,RS}}{R_{RS,LS}}-1\right]
(\mathcal{L}_{LR,T}^{ET}+\mathcal{L}_{LR,T}^{CAR})\Delta T_{LR}
+
\dfrac{R_{RS,RS}}{R_{RS,LS}}(\mathcal{L}^{QP}_{RS,T}\Delta T_{RS}+\mathcal{L}^{QP}_{LS,T}\Delta T_{LS}).
\end{align}

Next, we notice that 
\begin{align*}
R_{RS,RS}-R_{RS,LS}
=\dfrac{\mathcal{L}_{LR,\mu}^{ET}+\mathcal{L}_{LR,\mu}^{CAR}+2\mathcal{L}_{LL,\mu}^{DAR}+\mathcal{L}^{QP}_{LS,\mu}}{2eD}
-\dfrac{(\mathcal{L}_{LR,\mu}^{ET}-\mathcal{L}_{LR,\mu}^{CAR})}{2eD}
\end{align*}
which leads to
\begin{align*}
R_{RS,RS}-R_{RS,LS}
=\dfrac{2\mathcal{L}_{LR,\mu}^{CAR}+2\mathcal{L}_{LL,\mu}^{DAR}+\mathcal{L}^{QP}_{LS,\mu}}{2eD}
=
\dfrac{\mathcal{L}_{LR,\mu}^{CAR}+\mathcal{L}_{LL,\mu}^{DAR}+\mathcal{L}^{QP}_{LS,\mu}/2}{eD}
=
R_{RL,RS}.
\end{align*}

Substituting this expression back into Eq. \eqref{JRSD1:intermed:eq4}
\begin{align}\label{JRSD1:intermed:eq5}
J_{RS}=
\dfrac{\Delta\mu_{LS}}{eR_{RS,LS}}
+
\dfrac{R_{RL,RS}}{R_{RS,LS}}
(\mathcal{L}_{LR,T}^{ET}+\mathcal{L}_{LR,T}^{CAR})\Delta T_{LR}
+
\dfrac{R_{RS,RS}}{R_{RS,LS}}(\mathcal{L}^{QP}_{RS,T}\Delta T_{RS}+\mathcal{L}^{QP}_{LS,T}\Delta T_{LS}),
\end{align}
where we have used $\delta\mu_{S}=0$ such that $\Delta\mu_{LS}=\delta\mu_{L}-\delta\mu_{S}=\delta\mu_{L}$.

\subsubsection{Eq. (D2)}

Next, we proceed to demonstrate Eq. (D2). In this case, we start from Eq. \eqref{deltamuR:D1}
\begin{multline*}
\delta\mu_{R}=\dfrac{
(\mathcal{L}_{LR,\mu}^{ET}+\mathcal{L}_{LR,\mu}^{CAR}+2\mathcal{L}_{LL,\mu}^{DAR}+\mathcal{L}^{QP}_{LS,\mu})
}{\mathcal{L}_{LR,\mu}^{ET}-\mathcal{L}_{LR,\mu}^{CAR}}\delta\mu_{L}
+
\dfrac{
(\mathcal{L}_{LR,T}^{ET}+\mathcal{L}_{LR,T}^{CAR})\Delta T_{LR}
+
\mathcal{L}^{QP}_{LS,T}\Delta T_{LS}}{\mathcal{L}_{LR,\mu}^{ET}-\mathcal{L}_{LR,\mu}^{CAR}}
\end{multline*}
and express $\delta\mu_{L}$ in terms of $\delta\mu_{R}$. Thus, we have:
\begin{align*}
\delta\mu_{L}=\dfrac{\mathcal{L}_{LR,\mu}^{ET}-\mathcal{L}_{LR,\mu}^{CAR}}{(\mathcal{L}_{LR,\mu}^{ET}+\mathcal{L}_{LR,\mu}^{CAR}+2\mathcal{L}_{LL,\mu}^{DAR}+\mathcal{L}^{QP}_{LS,\mu})}\delta\mu_{R}
-
\dfrac{
(\mathcal{L}_{LR,T}^{ET}+\mathcal{L}_{LR,T}^{CAR})\Delta T_{LR}
+
\mathcal{L}^{QP}_{LS,T}\Delta T_{LS}}{\mathcal{L}_{LR,\mu}^{ET}+\mathcal{L}_{LR,\mu}^{CAR}+2\mathcal{L}_{LL,\mu}^{DAR}+\mathcal{L}^{QP}_{LS,\mu}},
\end{align*}
and since $\delta\mu_{S}=0$, we can rewrite both $\delta\mu_{L}$ and $\delta\mu_{R}$ as $\Delta\mu_{L}$ and $\Delta\mu_{R}$, respectively, which leads to:
\begin{align}\label{DeltamuLSRS}
\Delta\mu_{LS}=\dfrac{\mathcal{L}_{LR,\mu}^{ET}-\mathcal{L}_{LR,\mu}^{CAR}}{(\mathcal{L}_{LR,\mu}^{ET}+\mathcal{L}_{LR,\mu}^{CAR}+2\mathcal{L}_{LL,\mu}^{DAR}+\mathcal{L}^{QP}_{LS,\mu})}\Delta\mu_{RS}
-
\dfrac{
(\mathcal{L}_{LR,T}^{ET}+\mathcal{L}_{LR,T}^{CAR})\Delta T_{LR}
+
\mathcal{L}^{QP}_{LS,T}\Delta T_{LS}}{\mathcal{L}_{LR,\mu}^{ET}+\mathcal{L}_{LR,\mu}^{CAR}+2\mathcal{L}_{LL,\mu}^{DAR}+\mathcal{L}^{QP}_{LS,\mu}}.
\end{align}

Here it is convenient to rewrite Eq. \eqref{DeltamuLSRS} in terms of previous definitions:
\begin{align*}
\mathcal{L}_{LR,\mu}^{ET}+\mathcal{L}_{LR,\mu}^{CAR}+2\mathcal{L}_{LL,\mu}^{DAR}+\mathcal{L}^{QP}_{LS,\mu}
&=
2eDR_{RS,RS},
\\
\mathcal{L}_{LR,\mu}^{ET}-\mathcal{L}_{LR,\mu}^{CAR}=2eDR_{RS,LS,}
\end{align*}
which leads to
\begin{align}\label{DeltamuLSRS2}
\Delta\mu_{LS}=\dfrac{R_{RS,LS}}{R_{RS,RS}}\Delta\mu_{RS}
-
\dfrac{
(\mathcal{L}_{LR,T}^{ET}+\mathcal{L}_{LR,T}^{CAR})\Delta T_{LR}
+
\mathcal{L}^{QP}_{LS,T}\Delta T_{LS}}{2eDR_{RS,RS}}.
\end{align}


Substituting Eq. \eqref{DeltamuLSRS} into Eq. \eqref{JRSD1:intermed:eq5} we obtain,
\begin{multline*}
J_{RS}=
\dfrac{\Delta\mu_{LS}}{eR_{RS,LS}}
+
\dfrac{R_{RL,RS}}{R_{RS,LS}}
(\mathcal{L}_{LR,T}^{ET}+\mathcal{L}_{LR,T}^{CAR})\Delta T_{LR}
+
\dfrac{R_{RS,RS}}{R_{RS,LS}}(\mathcal{L}^{QP}_{RS,T}\Delta T_{RS}+\mathcal{L}^{QP}_{LS,T}\Delta T_{LS}),
\end{multline*}