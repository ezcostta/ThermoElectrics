\section{Appendix C}

Let's start from Eq. (9) from the paper, 
\begin{multline}\label{current}
J_{L}
=
\dfrac{2e}{h}\int dE~\mathcal{T}^{ET}(E)[f_{L}(E)-f_{R}(E)]
+
\dfrac{2e}{h}\int dE~\mathcal{T}^{DAR}(E)[f_{L}(E)-\tilde{f}_{L}(E)]
\\+
\dfrac{2e}{h}\int dE~\mathcal{T}^{CAR}(E)[f_{L}(E)-\tilde{f}_{R}(E)]
+
\dfrac{2e}{h}\int dE~\mathcal{T}^{QP}(E)[f_{L}(E)-f_{S}(E)]
\end{multline}
where we have used the definitions of Fermi-Dirac distributions functions: $f_{\alpha}=\{\exp[(E-\mu_{\alpha})/k_{B}T_{\alpha}]+1\}^{-1}$ and $\tilde{f}_{\alpha}=1-f_{\alpha}(-E)=\{\exp[(E+\mu_{\alpha})/k_{B}T_{\alpha}]+1\}^{-1}$ for electrons and holes, respectively.


\subsection{Derivation of thermoelectric properties}


Let's assume a general Fermi function for a given lead $\alpha$ characterized by a local chemical potential $\mu_{\alpha}$ and temperature $T_{\alpha}$ such that the corresponding equilibrium quantities are $\mu$ and $T$. In this case the differences between these quantities are expressed as
\begin{align*}
\delta\mu_{\alpha}&=\mu_{\alpha}-\mu
\\
\delta T_{\alpha}&=T_{\alpha}-T
\end{align*}
where $\theta_{\alpha}$ and $V_{\alpha}$ are temperature and voltage biases applied to the lead. The Fermi function is then given by:
\begin{align*}
f_{\alpha}=\left\{1+\exp\left[\dfrac{E-\mu_{\alpha}}{k_{B}T_{\alpha}}\right]\right\}^{-1},
\end{align*}
such that
\begin{align*}
f_{0}=\left\{1+\exp\left[\dfrac{E-\mu}{k_{B}T}\right]\right\}^{-1}.
\end{align*}

Before proceeding to the explicit expansion of the Fermi function, let's consider first some derivatives to be used in further calculations. For simplicity, we define:
\begin{align*}
x=\dfrac{E-\mu_{\alpha}}{k_{B}T_{\alpha}},
\end{align*}

\begin{align*}
\dfrac{\partial f_{\alpha}}{\partial \mu_{\alpha}}=\dfrac{\partial f_{\alpha}}{\partial x}\dfrac{\partial x}{\partial\mu_{\alpha}}=-\dfrac{1}{k_{B}T_{\alpha}}\dfrac{\partial f_{\alpha}}{\partial x},
\end{align*}

\begin{align*}
\dfrac{\partial f_{\alpha}}{\partial T_{\alpha}}=\dfrac{\partial f_{\alpha}}{\partial x}\dfrac{\partial x}{\partial T_{\alpha}}=-\dfrac{(E-\mu_{\alpha})}{k_{B}T^{2}_{\alpha}}\dfrac{\partial f_{\alpha}}{\partial x},
\end{align*}
and
\begin{align*}
\dfrac{\partial f_{\alpha}}{\partial E}=\dfrac{\partial f_{\alpha}}{\partial x}\dfrac{\partial x}{\partial E}=\dfrac{1}{k_{B}T_{\alpha}}\dfrac{\partial f_{\alpha}}{\partial x}
\end{align*}

The first two derivatives may expressed in terms of the last one as follows:
\begin{align}\label{first}
\dfrac{\partial f_{\alpha}}{\partial \mu_{\alpha}}=-\dfrac{\partial f_{\alpha}}{\partial E}
\end{align}
and
\begin{align}\label{second}
\dfrac{\partial f_{\alpha}}{\partial T_{\alpha}}=-\dfrac{(E-\mu_{\alpha})}{T_{\alpha}}\dfrac{\partial f_{\alpha}}{\partial E}.
\end{align}

A similar calculation may be done for the hole Fermi function, $\tilde{f}_{\alpha}$, i.e., 
\begin{align}\label{third}
\dfrac{\partial \tilde{f}_{\alpha}}{\partial \mu_{\alpha}}=+\dfrac{\partial \tilde{f}_{\alpha}}{\partial E}
\end{align}
and
\begin{align}\label{fourth}
\dfrac{\partial\tilde{f}_{\alpha}}{\partial T_{\alpha}}=-\dfrac{(E+\mu_{\alpha})}{T_{\alpha}}\dfrac{\partial \tilde{f}_{\alpha}}{\partial E}.
\end{align}

\subsection{Expanding the Fermi function}

We consider the expansion up to the first terms:
\begin{align*}
f_{\alpha}=f_{0}+\left.\dfrac{\partial f_{\alpha}}{\partial\mu_{\alpha}}\right]_{\mu_{\alpha}=\mu}(\mu_{\alpha}-\mu)
+
\left.\dfrac{\partial f_{\alpha}}{\partial T_{\alpha}}\right]_{T_{\alpha}=T}(T_{\alpha}-T)
\end{align*}
which can be written as follows,
\begin{align*}
f_{\alpha}=f_{0}-\dfrac{\partial f_{0}}{\partial E}(\mu_{\alpha}-\mu)
-
\dfrac{(E-\mu)}{T}\dfrac{\partial f_{0}}{\partial E}(T_{\alpha}-T)
\end{align*}

\begin{align*}
f_{\alpha}=f_{0}-\dfrac{\partial f_{0}}{\partial E}\delta\mu_{\alpha}
-
\dfrac{(E-\mu)}{T}\dfrac{\partial f_{0}}{\partial E}\delta T_{\alpha}.
\end{align*}

A similar expansion may be done for the hole Fermi function:
\begin{align*}
\tilde{f}_{\alpha}=f_{0}+\dfrac{\partial f_{0}}{\partial E}\delta\mu_{\alpha}
-
\dfrac{(E+\mu)}{T}\dfrac{\partial f_{0}}{\partial E}\delta T_{\alpha}.
\end{align*}

\subsection{Expanding the current}
By using these expansions on the fermi functinons, Eq. \eqref{current} may be expressed as follows:
\begin{multline}\label{current2}
J_{L}
=
\dfrac{2e}{h}\int dE~\mathcal{T}^{ET}(E)
\left[
f_{0}-\dfrac{\partial f_{0}}{\partial E}\delta\mu_{L}
-
\dfrac{(E-\mu)}{T}\dfrac{\partial f_{0}}{\partial E}\delta T_{L}
-\left(f_{0}-\dfrac{\partial f_{0}}{\partial E}\delta\mu_{R}
-
\dfrac{(E-\mu)}{T}\dfrac{\partial f_{0}}{\partial E}\delta T_{R}\right)
\right]
\\+
\dfrac{2e}{h}\int dE~\mathcal{T}^{DAR}(E)
\left[
f_{0}-\dfrac{\partial f_{0}}{\partial E}\delta\mu_{L}
-
\dfrac{(E-\mu)}{T}\dfrac{\partial f_{0}}{\partial E}\delta T_{L}
-\left(f_{0}+\dfrac{\partial f_{0}}{\partial E}\delta\mu_{L}
-
\dfrac{(E+\mu)}{T}\dfrac{\partial f_{0}}{\partial E}\delta T_{L}\right)
\right]
\\+
\dfrac{2e}{h}\int dE~\mathcal{T}^{CAR}(E)\left[
f_{0}-\dfrac{\partial f_{0}}{\partial E}\delta\mu_{L}
-
\dfrac{(E-\mu)}{T}\dfrac{\partial f_{0}}{\partial E}\delta T_{L}
-\left(f_{0}+\dfrac{\partial f_{0}}{\partial E}\delta\mu_{R}
-
\dfrac{(E+\mu)}{T}\dfrac{\partial f_{0}}{\partial E}\delta T_{R}\right)
\right]
\\+
\dfrac{2e}{h}\int dE~\mathcal{T}^{QP}(E)\left[
f_{0}-\dfrac{\partial f_{0}}{\partial E}\delta\mu_{L}
-
\dfrac{(E-\mu)}{T}\dfrac{\partial f_{0}}{\partial E}\delta T_{L}
-\left(f_{0}-\dfrac{\partial f_{0}}{\partial E}\delta\mu_{S}
-
\dfrac{(E-\mu)}{T}\dfrac{\partial f_{0}}{\partial E}\delta T_{S}\right)
\right]
\end{multline}
which leads to
\begin{multline}\label{current3}
J_{L}
=
\dfrac{2e}{h}\int dE~\mathcal{T}^{ET}(E)
\left[
\left(-\dfrac{\partial f_{0}}{\partial E}\right)(\delta\mu_{L}-\delta\mu_{R})
+
\dfrac{(E-\mu)}{T}\left(-\dfrac{\partial f_{0}}{\partial E}\right)(\delta T_{L}-\delta T_{R})
\right]
\\+
\dfrac{2e}{h}\int dE~\mathcal{T}^{DAR}(E)
\left[
2\left(-\dfrac{\partial f_{0}}{\partial E}\right)\delta\mu_{L}
-2
\dfrac{\mu}{T}\left(-\dfrac{\partial f_{0}}{\partial E}\right)\delta T_{L}
\right]
\\+
\dfrac{2e}{h}\int dE~\mathcal{T}^{CAR}(E)\left[
\left(-\dfrac{\partial f_{0}}{\partial E}\right)(\delta\mu_{L}+\delta\mu_{R})
+
\dfrac{E}{T}\left(-\dfrac{\partial f_{0}}{\partial E}\right)(\delta T_{L}-\delta T_{R})
-
\dfrac{\mu}{T}\left(-\dfrac{\partial f_{0}}{\partial E}\right)(\delta T_{L}+\delta T_{R})
\right]
\\+\dfrac{2e}{h}\int dE~\mathcal{T}^{QP}(E)
\left[
\left(-\dfrac{\partial f_{0}}{\partial E}\right)(\delta\mu_{L}-\delta\mu_{S})
+
\dfrac{(E-\mu)}{T}\left(-\dfrac{\partial f_{0}}{\partial E}\right)(\delta T_{L}-\delta T_{S})
\right].
\end{multline}

We rewrite the term corresponding to the $CAR$ in a different way, in order to make the definitions easier to understand, thus we have:
\begin{multline*}
J_{L}
=
\dfrac{2e}{h}\int dE~\mathcal{T}^{ET}(E)
\left[
\left(-\dfrac{\partial f_{0}}{\partial E}\right)(\delta\mu_{L}-\delta\mu_{R})
+
\dfrac{(E-\mu)}{T}\left(-\dfrac{\partial f_{0}}{\partial E}\right)(\delta T_{L}-\delta T_{R})
\right]
\\+
\dfrac{2e}{h}\int dE~\mathcal{T}^{DAR}(E)
\left[
2\left(-\dfrac{\partial f_{0}}{\partial E}\right)\delta\mu_{L}
-2
\dfrac{\mu}{T}\left(-\dfrac{\partial f_{0}}{\partial E}\right)\delta T_{L}
\right]
\\+
\dfrac{2e}{h}\int dE~\mathcal{T}^{CAR}(E)\left[
\left(-\dfrac{\partial f_{0}}{\partial E}\right)(\delta\mu_{L}+\delta\mu_{R})
+
\dfrac{(E-\mu)}{T}\left(-\dfrac{\partial f_{0}}{\partial E}\right)(\delta T_{L}-\delta T_{R})
\right.\\\left.+
\dfrac{\mu}{T}\left(-\dfrac{\partial f_{0}}{\partial E}\right)(\delta T_{L}-\delta T_{R})
-
\dfrac{\mu}{T}\left(-\dfrac{\partial f_{0}}{\partial E}\right)(\delta T_{L}+\delta T_{R})
\right]
\\+\dfrac{2e}{h}\int dE~\mathcal{T}^{QP}(E)
\left[
\left(-\dfrac{\partial f_{0}}{\partial E}\right)(\delta\mu_{L}-\delta\mu_{S})
+
\dfrac{(E-\mu)}{T}\left(-\dfrac{\partial f_{0}}{\partial E}\right)(\delta T_{L}-\delta T_{S})
\right],
\end{multline*}
which may be simplified to the form:
\begin{multline}\label{current6}
J_{L}
=
\dfrac{2e}{h}\int dE~\mathcal{T}^{ET}(E)
\left[
\left(-\dfrac{\partial f_{0}}{\partial E}\right)(\delta\mu_{L}-\delta\mu_{R})
+
\dfrac{(E-\mu)}{T}\left(-\dfrac{\partial f_{0}}{\partial E}\right)(\delta T_{L}-\delta T_{R})
\right]
\\+
\dfrac{2e}{h}\int dE~\mathcal{T}^{DAR}(E)
\left[
2\left(-\dfrac{\partial f_{0}}{\partial E}\right)\delta\mu_{L}
-2
\dfrac{\mu}{T}\left(-\dfrac{\partial f_{0}}{\partial E}\right)\delta T_{L}
\right]
\\+
\dfrac{2e}{h}\int dE~\mathcal{T}^{CAR}(E)\left[
\left(-\dfrac{\partial f_{0}}{\partial E}\right)(\delta\mu_{L}+\delta\mu_{R})
+
\dfrac{(E-\mu)}{T}\left(-\dfrac{\partial f_{0}}{\partial E}\right)(\delta T_{L}-\delta T_{R})
\right.\\\left.-2
\dfrac{\mu}{T}\left(-\dfrac{\partial f_{0}}{\partial E}\right)\delta T_{R}
\right]
\\+\dfrac{2e}{h}\int dE~\mathcal{T}^{QP}(E)
\left[
\left(-\dfrac{\partial f_{0}}{\partial E}\right)(\delta\mu_{L}-\delta\mu_{S})
+
\dfrac{(E-\mu)}{T}\left(-\dfrac{\partial f_{0}}{\partial E}\right)(\delta T_{L}-\delta T_{S})
\right],
\end{multline}



In order to further simplify the above expression, we use the following definitions:
\begin{align}\label{L:chemical}
\mathcal{L}_{\alpha\beta,\mu}^{\kappa}&=\dfrac{2e}{h}\int dE~\mathcal{T}^{\kappa}(E)
\left(-\dfrac{\partial f_{0}}{\partial E}\right),
\\\label{L:temperature}
\mathcal{L}_{\alpha\beta,T}^{\kappa}&=\dfrac{2e}{h}\int dE~\dfrac{(E-\mu)}{T}\mathcal{T}^{\kappa}(E)
\left(-\dfrac{\partial f_{0}}{\partial E}\right),
\end{align}
with $\alpha,\beta=\{L,R,S\}$ and $\kappa=\{QP,ET,CAR,DAR\}$.

By using Eqs. \eqref{L:chemical} and \eqref{L:temperature} into Eq. \eqref{current6}, we obtain:
\begin{multline*}
J_{L}=
\mathcal{L}_{LR,\mu}^{ET}(\delta\mu_{L}-\delta\mu_{R})
+
\mathcal{L}_{LR,T}^{ET}(\delta T_{L}-\delta T_{R})
+
2\mathcal{L}_{LL,\mu}^{DAR}\left(\delta\mu_{L}-\dfrac{\mu}{T}\delta T_{L}\right)
+
\mathcal{L}_{LR,\mu}^{CAR}(\delta\mu_{L}+\delta\mu_{R})
\\
+
\mathcal{L}_{LR,T}^{CAR}(\delta T_{L}-\delta T_{R})
-
2\dfrac{\mu}{T}\mathcal{L}_{LR,\mu}^{CAR}\delta T_{R}
+
\mathcal{L}^{QP}_{LS,\mu}(\delta\mu_{L}-\delta\mu_{S})
+
\mathcal{L}^{QP}_{LS,T}(\delta T_{L}-\delta T_{S})
\end{multline*}
which leads to
\begin{multline*}
J_{L}=
\mathcal{L}_{LR,\mu}^{ET}(\delta\mu_{L}-\delta\mu_{R})
+
\mathcal{L}_{LR,T}^{ET}(\delta T_{L}-\delta T_{R})
+
2\mathcal{L}_{LL,\mu}^{DAR}\left(\delta\mu_{L}-\dfrac{\mu}{T}\delta T_{L}\right)
\\+
\mathcal{L}_{LR,\mu}^{CAR}\left(\delta\mu_{L}-\dfrac{\mu}{T}\delta T_{R}+\delta\mu_{R}-\dfrac{\mu}{T}\delta T_{R}\right)
+
\mathcal{L}_{LR,T}^{CAR}(\delta T_{L}-\delta T_{R})
\\+
\mathcal{L}^{QP}_{LS,\mu}(\delta\mu_{L}-\delta\mu_{S})
+
\mathcal{L}^{QP}_{LS,T}(\delta T_{L}-\delta T_{S}),
\end{multline*}
and by grouping similar terms, we obtain:
\begin{multline}\label{JL:general}
J_{L}=
\mathcal{L}_{LR,\mu}^{ET}(\delta\mu_{L}-\delta\mu_{R})
+
(\mathcal{L}_{LR,T}^{ET}+\mathcal{L}_{LR,T}^{CAR})(\delta T_{L}-\delta T_{R})
+
2\mathcal{L}_{LL,\mu}^{DAR}\left(\delta\mu_{L}-\dfrac{\mu}{T}\delta T_{L}\right)
\\+
\mathcal{L}_{LR,\mu}^{CAR}\left(\delta\mu_{L}-\dfrac{\mu}{T}\delta T_{R}+\delta\mu_{R}-\dfrac{\mu}{T}\delta T_{R}\right)
\\+
\mathcal{L}^{QP}_{LS,\mu}(\delta\mu_{L}-\delta\mu_{S})
+
\mathcal{L}^{QP}_{LS,T}(\delta T_{L}-\delta T_{S}).
\end{multline}

\subsection{Setting the reference chemical potencial to zero: $\mu=0$}

By using the chemical potencial reference to zero, we obtain a simplified version of Eq. \eqref{JL:general}. In this way, we have:
\begin{multline}\label{JL:general:new}
J_{L}=
\mathcal{L}_{LR,\mu}^{ET}(\delta\mu_{L}-\delta\mu_{R})+
\mathcal{L}_{LR,\mu}^{CAR}\left(\delta\mu_{L}+\delta\mu_{R}\right)
+
(\mathcal{L}_{LR,T}^{ET}+\mathcal{L}_{LR,T}^{CAR})(\delta T_{L}-\delta T_{R})
+
2\mathcal{L}_{LL,\mu}^{DAR}\delta\mu_{L}
\\+
\mathcal{L}^{QP}_{LS,\mu}(\delta\mu_{L}-\delta\mu_{S})
+
\mathcal{L}^{QP}_{LS,T}(\delta T_{L}-\delta T_{S}).
\end{multline}
where we have definitions, 
\begin{align}\label{L:chemical:new}
\mathcal{L}_{\alpha\beta,\mu}^{\kappa}&=\dfrac{2e}{h}\int dE~\mathcal{T}^{\kappa}(E)
\left(-\dfrac{\partial f_{0}}{\partial E}\right),
\\\label{L:temperature:new}
\mathcal{L}_{\alpha\beta,T}^{\kappa}&=\dfrac{2e}{hT}\int dE~E\mathcal{T}^{\kappa}(E)
\left(-\dfrac{\partial f_{0}}{\partial E}\right).
\end{align}

We also have the current from the right lead which is obtained by performing the exchange $L\leftrightarrow R$ in Eq. \eqref{JL:general:new}. In this way, we have:
\begin{multline}\label{JR:general}
J_{R}=
\mathcal{L}_{RL,\mu}^{ET}(\delta\mu_{R}-\delta\mu_{L})+
\mathcal{L}_{RL,\mu}^{CAR}\left(\delta\mu_{R}+\delta\mu_{L}\right)
+
(\mathcal{L}_{RL,T}^{ET}+\mathcal{L}_{RL,T}^{CAR})(\delta T_{R}-\delta T_{L})
+
2\mathcal{L}_{RR,\mu}^{DAR}\delta\mu_{R}
\\+
\mathcal{L}^{QP}_{RS,\mu}(\delta\mu_{R}-\delta\mu_{S})
+
\mathcal{L}^{QP}_{RS,T}(\delta T_{R}-\delta T_{S}).
\end{multline}

We can also determine the current flowing into the $S$ lead by using current conservation: $J_{S}=-J_{L}-J_{R}$ which leads to: 
\begin{multline}\label{JS:general}
-J_{S}=
2\mathcal{L}_{LR,\mu}^{CAR}\left(\delta\mu_{L}+\delta\mu_{R}\right)
+
2\mathcal{L}_{LL,\mu}^{DAR}\delta\mu_{L}
+
2\mathcal{L}_{RR,\mu}^{DAR}\delta\mu_{R}
\\+
\mathcal{L}^{QP}_{LS,\mu}(\delta\mu_{L}-\delta\mu_{S})
+
\mathcal{L}^{QP}_{LS,T}(\delta T_{L}-\delta T_{S})
+
\mathcal{L}^{QP}_{RS,\mu}(\delta\mu_{R}-\delta\mu_{S})
+
\mathcal{L}^{QP}_{RS,T}(\delta T_{R}-\delta T_{S}).
\end{multline}

\subsection{Reproducing the equations of appendix C}

We consider the scenario in which the $L$ lead will be the voltage probe under isothermal conditions. In addition, we also set the voltage at $S$ equal to zero, $\delta\mu_{S}=0$. In this way, by using Eq. \eqref{JL:general:new}, we have:
\begin{align*}
0=
\mathcal{L}_{LR,\mu}^{ET}(\delta\mu_{L}-\delta\mu_{R})+
\mathcal{L}_{LR,\mu}^{CAR}\left(\delta\mu_{L}+\delta\mu_{R}\right)
+
2\mathcal{L}_{LL,\mu}^{DAR}\delta\mu_{L}
+
\mathcal{L}^{QP}_{LS,\mu}\delta\mu_{L}
\end{align*}
which leads to
\begin{align*}
0=
(\mathcal{L}_{LR,\mu}^{CAR}-\mathcal{L}_{LR,\mu}^{ET})\delta\mu_{R}+
(\mathcal{L}_{LR,\mu}^{ET}+\mathcal{L}_{LR,\mu}^{CAR}+2\mathcal{L}_{LL,\mu}^{DAR}
+
\mathcal{L}^{QP}_{LS,\mu})\delta\mu_{L}
\end{align*}
thus, 
\begin{align}\label{delta:mRmL}
\delta\mu_{R}=\dfrac{(\mathcal{L}_{LR,\mu}^{ET}+\mathcal{L}_{LR,\mu}^{CAR}+2\mathcal{L}_{LL,\mu}^{DAR}
+
\mathcal{L}^{QP}_{LS,\mu})}{\mathcal{L}_{LR,\mu}^{ET}-\mathcal{L}_{LR,\mu}^{CAR}}\delta\mu_{L}.
\end{align}


In order to derive Eq. (C1), we use the isothermal version of Eq. \eqref{JR:general}
\begin{align*}
J_{R}=J_{RS}=
(-\mathcal{L}_{RL,\mu}^{ET}+
\mathcal{L}_{RL,\mu}^{CAR})\delta\mu_{L}
+
(\mathcal{L}_{RL,\mu}^{ET}+\mathcal{L}_{RL,\mu}^{CAR}+2\mathcal{L}_{RR,\mu}^{DAR}
+
\mathcal{L}^{QP}_{RS,\mu})\delta\mu_{R}
\end{align*}
and by substituting Eq. \eqref{delta:mRmL} we have:
\begin{multline*}
J_{RS}=
(-\mathcal{L}_{RL,\mu}^{ET}+
\mathcal{L}_{RL,\mu}^{CAR})\delta\mu_{L}
\\+
(\mathcal{L}_{RL,\mu}^{ET}+\mathcal{L}_{RL,\mu}^{CAR}+2\mathcal{L}_{RR,\mu}^{DAR}
+
\mathcal{L}^{QP}_{RS,\mu})\dfrac{(\mathcal{L}_{LR,\mu}^{ET}+\mathcal{L}_{LR,\mu}^{CAR}+2\mathcal{L}_{LL,\mu}^{DAR}
+
\mathcal{L}^{QP}_{LS,\mu})}{\mathcal{L}_{LR,\mu}^{ET}-\mathcal{L}_{LR,\mu}^{CAR}}\delta\mu_{L}
\end{multline*}
which leads to 
\begin{multline*}
(\mathcal{L}_{LR,\mu}^{ET}-\mathcal{L}_{LR,\mu}^{CAR})\dfrac{J_{RS}}{\delta\mu_{L}}=
(-\mathcal{L}_{RL,\mu}^{ET}+
\mathcal{L}_{RL,\mu}^{CAR})(\mathcal{L}_{LR,\mu}^{ET}-\mathcal{L}_{LR,\mu}^{CAR})
\\+
(\mathcal{L}_{RL,\mu}^{ET}+\mathcal{L}_{RL,\mu}^{CAR}+2\mathcal{L}_{RR,\mu}^{DAR}
+
\mathcal{L}^{QP}_{RS,\mu})(\mathcal{L}_{LR,\mu}^{ET}+\mathcal{L}_{LR,\mu}^{CAR}+2\mathcal{L}_{LL,\mu}^{DAR}
+
\mathcal{L}^{QP}_{LS,\mu})
\end{multline*}
and by using the symmetry properties, we can further simplify the equation above
\begin{multline*}
(\mathcal{L}_{LR,\mu}^{ET}-\mathcal{L}_{LR,\mu}^{CAR})\dfrac{J_{RS}}{\delta\mu_{L}}=
-(\mathcal{L}_{LR,\mu}^{ET}-
\mathcal{L}_{LR,\mu}^{CAR})^{2}
\\+
(\mathcal{L}_{LR,\mu}^{ET}+\mathcal{L}_{LR,\mu}^{CAR}+2\mathcal{L}_{RR,\mu}^{DAR}
+
\mathcal{L}^{QP}_{RS,\mu})(\mathcal{L}_{LR,\mu}^{ET}+\mathcal{L}_{LR,\mu}^{CAR}+2\mathcal{L}_{LL,\mu}^{DAR}
+
\mathcal{L}^{QP}_{LS,\mu})
\end{multline*}



\begin{multline*}
(\mathcal{L}_{LR,\mu}^{ET}-\mathcal{L}_{LR,\mu}^{CAR})\dfrac{J_{RS}}{\delta\mu_{L}}=
-(\mathcal{L}_{LR,\mu}^{ET}-
\mathcal{L}_{LR,\mu}^{CAR})^{2}
+
(\mathcal{L}_{LR,\mu}^{ET}+\mathcal{L}_{LR,\mu}^{CAR})^{2}
+
(\mathcal{L}_{LR,\mu}^{ET}+\mathcal{L}_{LR,\mu}^{CAR})
(2\mathcal{L}_{LL,\mu}^{DAR}+\mathcal{L}^{QP}_{LS,\mu})
\\+
(2\mathcal{L}_{RR,\mu}^{DAR}+\mathcal{L}^{QP}_{RS,\mu})
(\mathcal{L}_{LR,\mu}^{ET}+\mathcal{L}_{LR,\mu}^{CAR})
+
(2\mathcal{L}_{RR,\mu}^{DAR}+\mathcal{L}^{QP}_{RS,\mu})
(2\mathcal{L}_{LL,\mu}^{DAR}+\mathcal{L}^{QP}_{LS,\mu})
\end{multline*}



\begin{multline*}
(\mathcal{L}_{LR,\mu}^{ET}-\mathcal{L}_{LR,\mu}^{CAR})\dfrac{J_{RS}}{\delta\mu_{L}}=
4\mathcal{L}_{LR,\mu}^{ET}\mathcal{L}_{LR,\mu}^{CAR}
+
2\mathcal{L}_{LL,\mu}^{DAR}\mathcal{L}_{LR,\mu}^{ET}+2\mathcal{L}_{LL,\mu}^{DAR}\mathcal{L}_{LR,\mu}^{CAR}
+
\mathcal{L}^{QP}_{LS,\mu}\mathcal{L}_{LR,\mu}^{ET}+\mathcal{L}^{QP}_{LS,\mu}\mathcal{L}_{LR,\mu}^{CAR}
\\+
2\mathcal{L}_{RR,\mu}^{DAR}\mathcal{L}_{LR,\mu}^{ET}+\mathcal{L}^{QP}_{RS,\mu}\mathcal{L}_{LR,\mu}^{ET}
+
2\mathcal{L}_{RR,\mu}^{DAR}\mathcal{L}_{LR,\mu}^{CAR}+\mathcal{L}^{QP}_{RS,\mu}\mathcal{L}_{LR,\mu}^{CAR}
\\+
4\mathcal{L}_{LL,\mu}^{DAR}\mathcal{L}_{RR,\mu}^{DAR}+2\mathcal{L}_{LL,\mu}^{DAR}\mathcal{L}^{QP}_{RS,\mu}
+
2\mathcal{L}_{RR,\mu}^{DAR}\mathcal{L}^{QP}_{LS,\mu}+\mathcal{L}^{QP}_{RS,\mu}\mathcal{L}^{QP}_{LS,\mu}
\end{multline*}
and performing the multiplications, we have:
\begin{multline*}
(\mathcal{L}_{LR,\mu}^{ET}-\mathcal{L}_{LR,\mu}^{CAR})\dfrac{J_{RS}}{\delta\mu_{L}}=
4\mathcal{L}_{LR,\mu}^{ET}\mathcal{L}_{LR,\mu}^{CAR}
+
2\mathcal{L}_{LL,\mu}^{DAR}\mathcal{L}_{LR,\mu}^{ET}
+
2\mathcal{L}_{LL,\mu}^{DAR}\mathcal{L}_{LR,\mu}^{CAR}
+
\mathcal{L}^{QP}_{LS,\mu}\mathcal{L}_{LR,\mu}^{ET}
+
\mathcal{L}^{QP}_{LS,\mu}\mathcal{L}_{LR,\mu}^{CAR}
\\+
2\mathcal{L}_{RR,\mu}^{DAR}\mathcal{L}_{LR,\mu}^{ET}
+
\mathcal{L}^{QP}_{RS,\mu}\mathcal{L}_{LR,\mu}^{ET}
+
2\mathcal{L}_{RR,\mu}^{DAR}\mathcal{L}_{LR,\mu}^{CAR}
+
\mathcal{L}^{QP}_{RS,\mu}\mathcal{L}_{LR,\mu}^{CAR}
\\+
4\mathcal{L}_{LL,\mu}^{DAR}\mathcal{L}_{RR,\mu}^{DAR}
+
2\mathcal{L}_{LL,\mu}^{DAR}\mathcal{L}^{QP}_{RS,\mu}
+
2\mathcal{L}_{RR,\mu}^{DAR}\mathcal{L}^{QP}_{LS,\mu}
+
\mathcal{L}^{QP}_{RS,\mu}\mathcal{L}^{QP}_{LS,\mu}.
\end{multline*}

Here we can group some terms as follows:
\begin{multline*}
(\mathcal{L}_{LR,\mu}^{ET}-\mathcal{L}_{LR,\mu}^{CAR})\dfrac{J_{RS}}{\delta\mu_{L}}=
\mathcal{L}_{LR,\mu}^{ET}(4\mathcal{L}_{LR,\mu}^{CAR}
+
2\mathcal{L}_{LL,\mu}^{DAR}+2\mathcal{L}_{RR,\mu}^{DAR}+\mathcal{L}^{QP}_{LS,\mu}+\mathcal{L}^{QP}_{RS,\mu})
\\+
\mathcal{L}_{LR,\mu}^{CAR}(2\mathcal{L}_{LL,\mu}^{DAR}+2\mathcal{L}_{RR,\mu}^{DAR}+\mathcal{L}^{QP}_{LS,\mu}+\mathcal{L}^{QP}_{RS,\mu})
\\+
4\mathcal{L}_{LL,\mu}^{DAR}\mathcal{L}_{RR,\mu}^{DAR}
+
2\mathcal{L}_{LL,\mu}^{DAR}\mathcal{L}^{QP}_{RS,\mu}
+
2\mathcal{L}_{RR,\mu}^{DAR}\mathcal{L}^{QP}_{LS,\mu}
+
\mathcal{L}^{QP}_{RS,\mu}\mathcal{L}^{QP}_{LS,\mu}.
\end{multline*}
or,
\begin{multline}\label{intermed:eq1}
(\mathcal{L}_{LR,\mu}^{ET}-\mathcal{L}_{LR,\mu}^{CAR})\dfrac{J_{RS}}{\delta\mu_{L}}=
\mathcal{L}_{LR,\mu}^{ET}(4\mathcal{L}_{LR,\mu}^{CAR}
+
2\mathcal{L}_{LL,\mu}^{DAR}+2\mathcal{L}_{RR,\mu}^{DAR})
+
\mathcal{L}_{LR,\mu}^{CAR}(2\mathcal{L}_{LL,\mu}^{DAR}+2\mathcal{L}_{RR,\mu}^{DAR})
+
4\mathcal{L}_{LL,\mu}^{DAR}\mathcal{L}_{RR,\mu}^{DAR}
\\+
2\mathcal{L}_{LL,\mu}^{DAR}\mathcal{L}^{QP}_{RS,\mu}
+
2\mathcal{L}_{RR,\mu}^{DAR}\mathcal{L}^{QP}_{LS,\mu}
+(\mathcal{L}^{QP}_{LS,\mu}+\mathcal{L}^{QP}_{RS,\mu})(\mathcal{L}_{LR,\mu}^{ET}+\mathcal{L}_{LR,\mu}^{CAR})+
\mathcal{L}^{QP}_{RS,\mu}\mathcal{L}^{QP}_{LS,\mu}.
\end{multline}


In order to comply with the definitions appearing on the paper, we define:
\begin{multline}\label{D:def}
D=
\mathcal{L}_{LR,\mu}^{ET}(2\mathcal{L}_{LR,\mu}^{CAR}
+
\mathcal{L}_{LL,\mu}^{DAR}+\mathcal{L}_{RR,\mu}^{DAR})
+
\mathcal{L}_{LR,\mu}^{CAR}(\mathcal{L}_{LL,\mu}^{DAR}+\mathcal{L}_{RR,\mu}^{DAR})
+
2\mathcal{L}_{LL,\mu}^{DAR}\mathcal{L}_{RR,\mu}^{DAR}
\\+
\mathcal{L}_{LL,\mu}^{DAR}\mathcal{L}^{QP}_{RS,\mu}
+
\mathcal{L}_{RR,\mu}^{DAR}\mathcal{L}^{QP}_{LS,\mu}
+
\dfrac{1}{2}(\mathcal{L}^{QP}_{LS,\mu}+\mathcal{L}^{QP}_{RS,\mu})(\mathcal{L}_{LR,\mu}^{ET}+\mathcal{L}_{LR,\mu}^{CAR})
+
\dfrac{1}{2}\mathcal{L}^{QP}_{RS,\mu}\mathcal{L}^{QP}_{LS,\mu}.
\end{multline}

with this definition, we can rewrite Eq. \eqref{intermed:eq1} as follows:
\begin{align}
(\mathcal{L}_{LR,\mu}^{ET}-\mathcal{L}_{LR,\mu}^{CAR})\dfrac{J_{RS}}{\delta\mu_{L}}=
2D
\end{align}
which leads to
\begin{align}\label{R:RS:LS:intermed}
\dfrac{\delta\mu_{L}}{eJ_{RS}}=\dfrac{\mathcal{L}_{LR,\mu}^{ET}-\mathcal{L}_{LR,\mu}^{CAR}}{2eD}.
\end{align}

Now, by considering $\delta\mu_{S}=0$, we can write:
\begin{align*}
\delta\mu_{L}=\delta\mu_{L}-\delta\mu_{S}=\Delta\mu_{LS}.
\end{align*}

Next, we can write:
\begin{align}\label{R:RS:LS:intermed}
R_{RS,LS}=\dfrac{\Delta\mu_{LS}}{eJ_{RS}}=\dfrac{\mathcal{L}_{LR,\mu}^{ET}-\mathcal{L}_{LR,\mu}^{CAR}}{2eD}.
\end{align}